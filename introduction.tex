\chapter{Introduction}
\thispagestyle{empty}
\label{chap:intro}
\chapterPage{This is the introduction to the thesis}


\section{Motivation}
\label{sec:intro_motiv}
This document is intended to be used as a template for PhD theses. While unofficial, it complies with the criteria set by the University of Luxembourg. It is based on the theses of many fine folks who obtained their PhD at the SerVal\footnote{\url{https://wwwen.uni.lu/snt/research/serval}} and TruX\footnote{\url{https://wwwen.uni.lu/snt/research/trux}} teams at the Interdisciplinary Centre for Security, Reliability, and Trust, University of Luxembourg.
The template can be downloaded at Overleaf\footnote{\url{https://www.overleaf.com/read/jjkrymcjpjkq}} and GitHub\footnote{\url{https://github.com/Trustworthy-Software/PhD-Thesis-Template}}.

The rest of this section shows examples for some of the typical features you'll want to use, in a not-so-serious fashion.
Some features have multiple versions. Check out \textit{thesis.cls} to "select" the version you prefer.


\section{Challenges}
\label{sec:intro_challenge}
\subsection{Some Challenges}


Cham et al.\cite{cham} discussed the temporal challenges of writing a thesis. Figure~\ref{fig:time} summarises their findings.

\begin{figure}[h]
    \centering
    \includegraphics[width=.8\textwidth]{pictures/introduction/phd1.jpg}
    \caption{Temporal Challenges}
    \label{fig:time}
\end{figure}


\subsection{More Challenges}
\begin{table}[h]
    \centering
    \begin{tabular}{|c|c|}\hline
         Demographic& Amount of tears  \\
         & (litres/week)\\\hline
         Babies&10\\
         Children&6.5\\
         Teenagers&0.5\\
         Young Adults&0.2\\
         PhD Students&35\\
         Adults&1.2\\
         Seniors&0.4\\\hline
    \end{tabular}
    \caption{The average amount of tears cried by selected demographics.}
    \label{tab:tears}
\end{table}
Table~\ref{tab:tears} showcases one of the biological challenges of being a PhD student,\note{Cedric}{This information may be inaccurate} hinting at the importance of regular personal well-being courses. 

\highlight{\textbf{Main finding:} Being a PhD student is tough.}
While doing a PhD and writing a thesis are generally highly demanding, we determined that this endeavour can be overcome by applying the following approach:\\
\begin{flushleft}
\code{
while drink("coffee")==True:\\
\hspace{1cm}do\_work()
}
\end{flushleft}
\vspace{1cm}



\section{Contributions}
\label{sec:intro_contrib}

\section{Roadmap}
\label{sec:intro_road}

